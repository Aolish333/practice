\section{Analysis of Overall and Key Points}
\noindent In the first part of the question, two key points should be considered in predicting trends of global languages: the total population change in a given language and the geographical distribution of the language. In the section on simulating changes in total population speaking a center language, we need to collect historical data on changes in the number of language speakers. Next, we determine the geographical distribution of languages. To simulate the geographical distribution of the population using different languages, we consider two key factors: the change in the number of people using a language in the world and the immigration situation. For the former, we can analyze the deviation of the population center in speaking the language by comparing the changes in the number of people in the same language but in different regions. For the latter, we establish a migration model to simulate immigration, which gives the geographical distribution changes of the language caused by the immigrants.
\par In the second part of the question, when locating an international office, the effectiveness of the local language should be taken into account, namely the usefulness of the language. To take an extreme example, if a Martian descends upon Earth, what language should a Martian learn to understand the Earth as much as possible, and English is the most suitable language for now. So we propose an evaluation model of the effectiveness of languages in the future, according to which we can rank the language based on its effectiveness and determine where international offices should be constructed on the ranking results.
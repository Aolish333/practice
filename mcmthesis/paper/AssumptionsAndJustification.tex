\section{Assumptions and Justification}

\begin{itemize} 
	
    \item \textbf{Do not consider unexpected factors such as war factors. }We assume that some local language will not disappear due to other unexpected factors such as war. Since the probability of such an incident is minimal, we do not consider it.
    
    \item \textbf{Suppose the data collected is reliable .}Since most of the data we collect comes from the United Nations statistics, we consider these data to be reliabl.
    
    \item \textbf{Suppose the population using lower-ranked languages will not surge.}With the total population as the ranking criterion, the top 20 languages have the potential to enter the top 10, and considering the lower-ranked languages thave small population base,we ignore them.
    
    \item \textbf{Ignores the impact of policy on language distribution.} Due to the great uncontrollability and unpredictability of government policy making, this model does not consider the influence of government policy on language distribution.
    
    \item \textbf{When considering the mode of population migration, government intervention in population movements is ignored.}The immigration policy implemented by a government is related to numerous factors such as the demographic structure of each stage of our country and international partnerships. As a result, there is a great deal of uncertainty in the policy. Therefore, simplifying the analysis does not consider the policy factors.
    
	
\end{itemize}
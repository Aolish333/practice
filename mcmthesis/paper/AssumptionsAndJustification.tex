\section{Assumptions and Justification}

\begin{itemize} 
	
    \item \textbf{Each month is 30 days in Drifting season, that is, the total number of days people can drifting is 180.}To simplify our model we assume 30 days per month, and this assumption is reasonable.
    
    \item \textbf{The tourists' choice of travel time complies with the Poisson distribution.}Since the choice of tourists for travel dates is unknown, the Poisson distribution is a good way to model this process and to facilitate our model building.
    
    \item \textbf{The demand for drifting is always greater than the supply.}which is a assumption based on the facts set by the context of the incident.
    
    \item \textbf{All travel teams are drifting during the day and their maximum daily drifting time is 8 hours.}According to Reference 1, we assume that the maximum daily drifting time for a tour group is 8 and the tour group can only go downstream.
    
    \item \textbf{The daily travel distance and rafting times chosen by the tour group for the same number of travel days are up to themselves.}This assumption is very realistic, river management companies will not be fixed tourist travel itinerary.
    
	\item \textbf{Assuming that for any tour group, once they have chosen the mode of transport, they will not change their means of transport for the entire journey. }We assume that the river management department to take into account the efficient use of the efficiency of the vessel, for any tour only to provide a means of transport.
	
	\item \textbf{The total number of camping sites unchanged.}According to Reference 1, we assume that there are 38 camps evenly distributed along the bank.
	
	%\begin{itemize} 
		%\item A nested item. 
		%\item[+] A ‘plus’ item. 
		%\item Another item. 
	%\end{itemize} 
	\item The total number of campsites remains unchanged, reference 1, and we assume that there are 38 camps evenly distributed along the bank. 
\end{itemize}


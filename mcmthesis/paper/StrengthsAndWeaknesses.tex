\section{Strengths and weaknesses}
\subsection{Strengths}
\begin{itemize}
	\item \textbf{Improve the existing model}
	\par Existing indicators of linguistic influence assessment model change and the evaluation of linguistic effectiveness that is more suitable for solving this problem are improved. To quantify the linguistic influence of the abstract description to a fraction of [0,1] and calculate the language score to provide a reasonable basis for the new international office location
	\item \textbf{Good flexibility}
	\par To study this problem, two main models are established: the non-equidistant GM (1,1) prediction model and the language performance evaluation model (LII)
	\par The former model can be used for other forecasting problems. The latter model can be extended to the evaluation problem. The model is more flexible and can be transformed into a solution to other practical problems
	\item \textbf{youdian}
	
	\item \textbf{youdian}
	
\end{itemize}

\subsection{weaknesses}

\begin{itemize}
	\item \textbf{Inaccuracy}
	\item \textbf{Simplifying assumptions}
	\item \textbf{Lack of data}
\end{itemize}
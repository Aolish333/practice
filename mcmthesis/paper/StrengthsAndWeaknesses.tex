\section{Strengths and weaknesses}
\subsection{Strengths}
\begin{itemize}
	\item \textbf{Improve the existing model}
\newline Existing indicators of linguistic influence assessment model change and the evaluation of linguistic effectiveness that is more suitable for solving this problem are improved. To quantify the linguistic influence of the abstract description to a fraction of [0,1] and calculate the language score to provide a reasonable basis for the new international office location
	\item \textbf{Good flexibility}
\newline To study this problem, two main models are established: the non-equidistant GM (1,1) prediction model and the language performance evaluation model (LII). The former model can be used for other forecasting problems. The latter model can be extended to the evaluation problem. The model is more flexible and can be transformed into a solution to other practical problems
	\item \textbf{youdian}
	
	\item \textbf{youdian}
	
\end{itemize}

\subsection{weaknesses}

\begin{itemize}
	\item \textbf{Inaccuracy} \newline
	In the first part of the forecast of the development trend of the global language, due to the time factor analysis did not use multivariate analysis, nor did it fully consider the background of the factors. Based on the actual data we found, we used the predictive model to predict the data, which is a little bit different from the actual situation
	\item \textbf{Simplifying assumptions}
	\newline In the establishment of a population migration model, the simplification assumption assumes that only the net migration population is considered, the gravitation analysis for the population movements is not enough, and the geographical distribution and proliferation of the second language caused by the actual population flows in terms of the number of refugees, economic factors and policy factors are not tapped.
	\item \textbf{Lack of data}
	\newline 
	The census and statistics of the global language population are extremely difficult and complex tasks. This question is required to collect a lot of data, the subject only gives a year of data, you must have nearly 10 years of data to make long-term forecasts. We collect less valid data, which leads to inaccurate results of the forecasting model.
\end{itemize}
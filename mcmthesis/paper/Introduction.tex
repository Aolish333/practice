\section{Introduction}

\subsection{Background}
\noindent
There are currently about 6909 languages in the world, and the total number and geographical distribution of each language is a necessary consideration for international organizations and economic development. The so-called language population refers to the native language of the population \upcite{wiki}. In the analysis of the number of native speakers of the world language, nearly half of the population is native speakers in the following 10 languages: Mandarin (incl. Standard Chinese), Spanish, English, Hindi, Arabic, Bengali, Portuguese, Russian, Punjabi, and Japanese.

However, many people use this as their second language. The total number is not determined solely by the number of native speakers,that can determine the total number of 0.4~0.8. For a language as a second language or a third language or even more, the number of people in order and the order of the mother tongue is not the same arrangement. Therefore, when analyzing the trend of global language use, we should not only consider the number of native speakers, but also the change of the number of  second and third languages as non-native speakers. 

Over time, the increase or decrease in the total number of uses of a language is influenced by a number of factors. These factors are broadly divided into policy factors: the official language of a government or the promotion of a language, educational factors: the language of school teaching, social factors: employment pressure, cultural factors: cultural diffusion and assimilation phenomenon, demographic factors: the country's demographic changes and migration led to population migration. At the same time, due to the rapid development of the global economy, the increase of international business and transnational corporations, economic factors can drive the influence of a country's language and thus the total number of people who use language.Now the internet is popular, the world is closely related, the use of communication media and the help of mobile software, such as accurate and rapid language translation and other network factors can also affect the development of language. These factors may have an impact on the trend of language development, but not just that. 
\subsection{Restatement of the problem}
\noindent
We are required to investigate trends of global languages,and provide a multinational service company with a new international office location plan.
\par We understand the problem as follows:

	\begin{itemize}
		\item
We are asked to set up a model to describe the distribution of language over time based on possible influencing factors.
		\item
We should predict how the number of native speakers and total speakers will change in the next 50 years and whether the top 10 native speakers and total speakers will be replaced by another language.
		 \item
Based on the world population growth and immigration patterns for the next 50 years, we need to determine whether the geographical distribution of the language will change during this period. If so, describe the change.
		 \item 
Provide international service companies with site selection plans for new offices and consider whether the programs will be different from the perspective of long-term and short-term.
		\item 
Given the changing nature of global communications, in order to reduce the number of new international offices, we are supposed to consider additional information and give further advice based on additional information.Finally, we were also asked to write a memorandum to the relevant department. 		 	 
	\end{itemize}



\subsection{Our works}
When predicting the of language, two key factors should be taken into account: the change of the total population in a given language and the geographical distribution of the language. These two factors determine the number and spatial distribution of language speakers, so we need to look at how these two factors change over time. The dominant language of the geographical area to be selected to construct a new International Office should have great international influence, more appropriately, its effectiveness. Therefore, we must make a quantitative assessment of the effectiveness of the language. Our work is completed in the following steps:

\begin{itemize}
	\item 
	Determine how the number of language speakers changes over time. Due to the large roughness and incompleteness of the existing statistical data, we established a non-equidistant gray prediction model. Applying this model,we eliminate the randomness of collected data and predict how the number of language speakers changes.
	\item Ascertain the transformation in the geographic distribution of the language speakers. We consider the change of geographical distribution of native and second languages. Define population concentration to determine the geographical transfer of the native speaker distribution center.We consider the mode of immigration to determine the spatial shift in the second language. Combining the above two parts we can get the trend of the mainstream languages and the geographical distribution of each language,and we can determine the language rankings and their changes in the next 50 years. 
	\item Evaluate the effectiveness of the language. We establish the evaluation model of the international influence of language in order to determine the effectiveness of the language. Based on the language's score given by the above model, the language area for the new international office is determined. Finally, we follow consideration of the rapid development of the Internet and global social media. We suppose that we can capture the penetration of regional Internet and that the multinational companies have web services.If so, We can regard the level of Internet penetration in a region as a guideline for the need to establish an office there. That is the higher the Internet penetration rate in a given area, the lower the need to have an office there, since the company can handle the business through its network channels
	
\end{itemize}


\subsection{Conceptual Analysis}

\begin{itemize}
	\item \textbf{Native language and second language}
	\par In 1951, UNESCO convened a meeting on the mother tongue in Paris and defined the mother tongue as follows: "Mother tongue refers to the language a person learns from an early age and is usually a natural tool for his thinking and communication.The second language refers to a person who, besides the first language, learns and understands the second language, often as an auxiliary language and a common language.
	\item \textbf{Second language and foreign language}
	\par The difference between a second language and a foreign language often depends on education. Although neither is born speaker, the former refers to the language that possesses context during the learning phase, the latter usually refers to a language that lacks context during the learning phase.
	For example, in China, English has been ever more common, and attractions, media and airports have English. However, China does not entirely popularize English context. Therefore, English continues to be a foreign language, not a second language. As in Egypt, English is also a foreign language.
\end{itemize}

\begin{abstract}
\noindent	
With the rise of rafting and the growing economic benefits it generates, river management urgently needs a viable solution to capitalize on river resources. Therefore, it is very necessary to make reasonable arrangements for the travel itinerary.
\par Firstly, we first define the river carrying capacity as the maximum number of river reception tours each year. To allow the river to accommodate as many tours as possible we set the first optimization goal, which is to maximize the number of trips per year. Secondly, we clarify the specific meaning of the contact of tour groups, followed by the construction of the second optimization goal, that is, to minimize the number of tour groups contacts. We then define the unit camping matrix, the camping matrix and the total camping matrix.Next,we take the use of camping sites for each tour, the use of arbitrary camping sites throughout the drift season, the use of all camping sites at any one night, and the number difference for two consecutive nights of camping in any group of tourists as the constraints. Considering the complexity of the solution, we weaken the second optimization target into a constraint condition, and it can relax or tighten the constraint to adjust the scheme according to the concrete effect of the scheme implementation.
\par Finally, we use JAVA as a programming tool to find X = 918 for Y = 38, which means 918 trips can be made throughout the rafting season when 38 campsites are evenly distributed along the Great River. In other words, the 918 specific travel plans that are available through our model.Then, the schedule of each plan is presented in the appendix At the end of this paper, we analyze the sensitivity of the model. By changing the maximum drifting time per day to 7$h$,7.5$h$,8.5$h$,9$h$ , we find that the change in the carrying capacity of the river is $ \pm 9{\rm{\% }}$ , which shows that our model has better sensitivity. And by increasing the number of camps, we find that the carrying capacity of the river is generally increased by exponential, which meets the objective law. 
% \lipsum[1]
\begin{keywords}
 bi-objective; optimization;0-1 matrix ; Poisson distribution
\end{keywords}
\end{abstract}
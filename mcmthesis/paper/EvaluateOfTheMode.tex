\section{Sensitivity analysis of the model}
\noindent
In the language performance evaluation model, the weight of each indicator is defined with reference to relevant information. For this we can compare the last language rankings with big changes by changing the index weights. Discuss two new ways to specify weights:

\begin{itemize}
	\item 5 indicators equally, each 0.2.
	\item Considering the actual situation, the influence of the area and the international use on the language is weak, taking 0.1.The remaining three factors bisect 0.8.
\end{itemize}


% Table generated by Excel2LaTeX from sheet 'Sheet1'
\begin{table}[H]
	\centering
	\caption{Indicator weighting table for three different scenarios}
	\scalebox{0.87}[0.87]{%
	\begin{tabular}{cp{5em}p{5em}p{5em}p{5em}p{6em}}
		\toprule
		 & Area & Economy strength & Language speakers & Knowledge and media &International use\\
		\midrule
		original plan & 0.225 & 0.225 & 0.225 & 0.225 & 0.1 \\
		\midrule
		Plan 1 & 0.25  & 0.25  & 0.25  & 0.25  & 0.25 \\
		\midrule
		Plan 2 & 0.1   & 0.27  & 0.27  & 0.27  & 0.1 \\
		\bottomrule
	\end{tabular}%
}
	\label{tab:addlabel}%
\end{table}%


Use the score calculation method in 5.2.1 to calculate the rankings of the language performance scores of three different weights. The ranking table is given in the appendix.It can be seen that the rankings of the first seven languages do not change, and the rankings of the latter four languages are slightly changed. Therefore, we can draw a conclusion that the language performance evaluation model we established has good stability.
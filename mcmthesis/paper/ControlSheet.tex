 %==========================================以上不要动=================================================
  \pagestyle{empty}%
  \null%
  \vspace*{-6pc}%
  \begin{center}
  \begingroup
  \setlength{\parindent}{0pt}
     \begin{minipage}{0.28\linewidth}
      For office use only\\[4pt]
      \makebox[0.15\linewidth][l]{T1}\rule[-2pt]{0.85\linewidth}{0.5pt}\\[4pt]
      \makebox[0.15\linewidth][l]{T2}\rule[-2pt]{0.85\linewidth}{0.5pt}\\[4pt]
      \makebox[0.15\linewidth][l]{T3}\rule[-2pt]{0.85\linewidth}{0.5pt}\\[4pt]
      \makebox[0.15\linewidth][l]{T4}\rule[-2pt]{0.85\linewidth}{0.5pt}
     \end{minipage}%
     \begin{minipage}{0.44\linewidth}
      \centering
      Team Control Number\\[0.7pc]
      {\LARGE\textbf{\newteamnumber}}\\[1.8pc]
      Problem Chosen\\[0.7pc]
      {\Huge\textbf{\newproblemchosen}}
     \end{minipage}%
     \begin{minipage}{0.28\linewidth}
      For office use only\\[4pt]
      \makebox[0.15\linewidth][l]{F1}\rule[-2pt]{0.85\linewidth}{0.5pt}\\[4pt]
      \makebox[0.15\linewidth][l]{F2}\rule[-2pt]{0.85\linewidth}{0.5pt}\\[4pt]
      \makebox[0.15\linewidth][l]{F3}\rule[-2pt]{0.85\linewidth}{0.5pt}\\[4pt]
      \makebox[0.15\linewidth][l]{F4}\rule[-2pt]{0.85\linewidth}{0.5pt}
     \end{minipage}\par
     \vskip 10pt
  \rule{\linewidth}{0.5pt}\par
  \vskip 10pt
  \textbf{{\Large\the\year}\\%
%  Mathematical Contest in Modeling (MCM/ICM) Summary Sheet}%
  MCM/ICM \\
  Summary Sheet}%
  \par
  \endgroup
  \vskip 10pt
  \normalfont \Large \ourtitle \par
  \centering {\normalsize{\textbf{summary}}}
  \end{center}\par
%==========================================以上不要动=================================================







% ============================================摘要======================================================
\noindent	
With the rise of rafting and the growing economic benefits it generates,it is very necessary to make reasonable arrangements for the travel itinerary.
\par In this paper, we propose a two-objective optimization model.To allow the river to accommodate as many tours as possible, we set the first optimization goal, which is to maximize the number of trips per year. Secondly, we clarify the specific meaning of the contact of tour groups, followed by the construction of the second optimization goal, that is, to minimize the number of tour groups contacts. We then define the unit camping matrix, the camping matrix and the total camping matrix.Next,we take the use of camping sites for each tour, the use of arbitrary camping sites throughout the drift season, the use of all camping sites at any one night, and the number difference for two consecutive nights of camping in any group of tourists as the constraints. Considering the complexity of the solution, we weaken the second optimization goal into a constraint condition, and it can relax or tighten the constraints to adjust the scheme according to the concrete effect of the scheme implementation.
\par At the end of this article, we use $JAVA$ as a programming tool to find X = 918 for Y = 38, which means 918 trips can be made throughout the rafting season when there are 38 campsites.Then, the schedule of each plan is presented in the appendix.Finally, we analyze the sensitivity of the model. By changing the maximum drifting time per day to 7$h$,7.5$h$,8.5$h$,9$h$ ,we find that the change in the carrying capacity of the river is $ \pm 9{\rm{\% }}$ , which shows our model has better sensitivity. And by increasing the number of camps, we find that the carrying capacity of the river is generally increased by exponential.In addition, we  find that with the increase of the number of camping sites the optimal model produced when the number of contacts between groups increases correspondingly, which indicates that the maximum utilization of the camp and the minimum number of contacts are the optimization objectives of a pair of contradictions. Unfortunately, we do not have enough time,otherwise we will quantify the relationship between the maximum number of trips and the number of contacts with the travel team so as to find a balance point between the two based on the actual situation.
% ============================================摘要======================================================
\\
% ============================================关键字======================================================
\par
\noindent
\textbf{Key words:} bi-objective; optimization; 0-1 matrix ; Poisson distribution










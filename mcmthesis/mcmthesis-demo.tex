% =================================快捷键及注意事项================================
%1. 注释
%"Ctrl" + "T"
%2. 去除注释
%“Ctrl” + "U"
%*************************** 用github 提交时,只提交你修改的相关文件。日志等不提交。
\documentclass{mcmthesis}
% =================================原始控制页模板设置=======================================
\mcmsetup{CTeX = true,   % 使用 CTeX 套装时,设置为 true
        tcn = 89760, problem = B,
        sheet = false, titleinsheet = true, keywordsinsheet = true,
        titlepage = false, abstract = true}
\usepackage{palatino}
\usepackage{caption} % 图片
\usepackage{subfigure} % 图片
\usepackage{lipsum} % 文本
\usepackage{booktabs} % 表格
\usepackage{makecell} % 表格线段的粗细
\usepackage{supertabular} % 多页表格
\usepackage{float} % 去除图片浮动
\usepackage[numbers,sort&compress]{natbib} % 引入多个参考文献
\setlength{\parskip}{0.2em} %设置行距
\usepackage{indentfirst}
\usepackage{slashbox}
\newcommand{\upcite}[1]{\textsuperscript{\textsuperscript{\cite{#1}}}} % 引用上标
\setlength{\parindent}{2em}
\bibliographystyle{plain}
% =================================自制控制页设置=======================================
\newcommand{\newteamnumber}{89760} 
\newcommand{\newproblemchosen}{B} 
\newcommand{\ourtitle}{The future of global languages} 
% =================================自制控制页设置=======================================
\title{The future of global languages} % 标题
\author{\small \href{http://www.latexstudio.net/}
  {\graphics[width=7cm]{mcmthesis-logo}}}
\date{\today}

\begin{document}
% =================================正文设置=======================================
\begin{abstract}
\noindent If an alien came after 50 years, what language should it learn to understand human civilization more fully? The research findings in this article can give an answer.
\par In this article, we predict trends in the mainstream languages of the world today and determine the address of a new international office for a multinational corporation.
\par The first stage: from the total number of language population and geographical distribution of language to predict the trend of global language development.For the former, we assume that there is no sudden factor to change the number of language speakers in a certain region drastically. Since there is a missing value in the collected data, a non-equidistant GM (1,1) prediction model is established to predict the total number of language speakers.Then analyses the trend of the total population in the first 20 languages.For the latter, consider the impact of the two factors of geographical population growth and population migration on geographical distribution of languages. 
\par The second stage: the establishment of language performance evaluation model (LII model) to determine the location of the new international offices.The LII model considers the specific 15 indicators of a language group in five major areas: Area, Economic strength, Language speakers, Knowledge and communication, International use. Firstly the indicator value is quantified to [0, 1], then weighted to obtain the language performance score and ranked.Analyze change of rankings of language performance score every 10 years in the future, and find that there is no change in the categories of the top 6 languages in terms of short-term or long-term.We assume that the offices are usually established in the capital of the country where the first language is the mother tongue, so six new offices are located: London, England / Paris, France / Madrid, Spain / Dammam, Saudi Arabia / Moscow, Russia / Berlin, Germany.Finally, considering the nature of the change in the global communications (geographically distant Internet connections and language barrier with powerful translation software), we think it is possible to set up an international office in each continent, that is, in addition to the United States and China , Confirming the establishment of four more new international offices. 	

\begin{keywords}

\end{keywords}
\end{abstract} % 模板自带的摘要
%\newgeometry{left=2.5cm,right=2.5cm}
 %==========================================以下不要动=================================================

  \pagestyle{empty}%
  \null%
  \vspace*{-6pc}%
  \begin{center}
  \begingroup
  \setlength{\parindent}{0pt}
     \begin{minipage}{0.28\linewidth}
      For office use only\\[4pt]
      \makebox[0.15\linewidth][l]{T1}\rule[-2pt]{0.85\linewidth}{0.5pt}\\[4pt]
      \makebox[0.15\linewidth][l]{T2}\rule[-2pt]{0.85\linewidth}{0.5pt}\\[4pt]
      \makebox[0.15\linewidth][l]{T3}\rule[-2pt]{0.85\linewidth}{0.5pt}\\[4pt]
      \makebox[0.15\linewidth][l]{T4}\rule[-2pt]{0.85\linewidth}{0.5pt}
     \end{minipage}%
     \begin{minipage}{0.44\linewidth}
      \centering
      Team Control Number\\[0.7pc]
      {\LARGE\textbf{\newteamnumber}}\\[1.8pc]
      Problem Chosen\\[0.7pc]
      {\Huge\textbf{\newproblemchosen}}
     \end{minipage}%
     \begin{minipage}{0.28\linewidth}
      For office use only\\[4pt]
      \makebox[0.15\linewidth][l]{F1}\rule[-2pt]{0.85\linewidth}{0.5pt}\\[4pt]
      \makebox[0.15\linewidth][l]{F2}\rule[-2pt]{0.85\linewidth}{0.5pt}\\[4pt]
      \makebox[0.15\linewidth][l]{F3}\rule[-2pt]{0.85\linewidth}{0.5pt}\\[4pt]
      \makebox[0.15\linewidth][l]{F4}\rule[-2pt]{0.85\linewidth}{0.5pt}
     \end{minipage}\par
     \vskip 10pt
  \rule{\linewidth}{0.5pt}\par
  \vskip 10pt
  \textbf{{\Large\the\year}\\%
%  Mathematical Contest in Modeling (MCM/ICM) Summary Sheet}%
  MCM/ICM \\
  Summary Sheet}%
  \par
  \endgroup
  \vskip 10pt
  \normalfont \Large \ourtitle \par
  \centering {\normalsize{\textbf{summary}}}
  \end{center}\par
%==========================================以上不要动=================================================

% ============================================摘要======================================================

















% ============================================摘要======================================================

% ============================================下面关键字======================================================
\par
\noindent
\textbf{Key words:}
 % 控制页及摘要 
\restoregeometry % 恢复原来的设置
\maketitle  % 模板自带:添加
\tableofcontents % 添加目录
\newpage % 段页
% =================================子文件夹设置====================================
% 在子文件夹下编写文件
\section{Introduction}
\subsection{Background}
\subsection{Restatement of the problem}
\subsection{Our works} % 介绍
\section{Analysis of Overall and Key Points }



\begin{equation} a^2 + b^2 = c^2 \end{equation} % 模型的分析
\section{Assumptions and Justification}

\begin{itemize} 
	
    \item \textbf{Each month is 30 days in Drifting season, that is, the total number of days people can drifting is 180.}To simplify our model we assume 30 days per month, and this assumption is reasonable.
    
    \item \textbf{The tourists' choice of travel time complies with the Poisson distribution.}Since the choice of tourists for travel dates is unknown, the Poisson distribution is a good way to model this process and to facilitate our model building.
    
    \item \textbf{The demand for drifting is always greater than the supply.}which is a assumption based on the facts set by the context of the incident.
    
    \item \textbf{All travel teams are drifting during the day and their maximum daily drifting time is 8 hours.}According to Reference 1, we assume that the maximum daily drifting time for a tour group is 8 and the tour group can only go downstream.
    
    \item \textbf{The daily travel distance and rafting times chosen by the tour group for the same number of travel days are up to themselves.}This assumption is very realistic, river management companies will not be fixed tourist travel itinerary.
    
	\item \textbf{Assuming that for any tour group, once they have chosen the mode of transport, they will not change their means of transport for the entire journey. }We assume that the river management department to take into account the efficient use of the efficiency of the vessel, for any tour only to provide a means of transport.
	
	\item \textbf{The total number of camping sites unchanged.}According to Reference 1, we assume that there are 38 camps evenly distributed along the bank.
	
	%\begin{itemize} 
		%\item A nested item. 
		%\item[+] A ‘plus’ item. 
		%\item Another item. 
	%\end{itemize} 
	\item The total number of campsites remains unchanged, reference 1, and we assume that there are 38 camps evenly distributed along the bank. 
\end{itemize}

  % 假设
\section{Symbols and Definitions }
In the section, we use some symbols for constructing the model as follows: 



%\begin{center}
%\tablefirsthead{%
%	\hline
%	\multicolumn{1}{|c}{\tbsp Symbolr} &
%	\multicolumn{1}{c}{Definition} & \\
%	\hline}
%\tablehead{%
%	\hline
%	\multicolumn{2}{|l|}{\small\sl continued from previous page}\\
%	\hline
%	\multicolumn{1}{|c}{\tbsp Symbolr} &
%	\multicolumn{1}{c}{Definition} & \\
%	\hline}
%\tabletail{%
%	\hline
%	\multicolumn{2}{|r|}{\small\sl continued on next page}\\
%	\hline}
%\tablelasttail{\hline}
%\bottomcaption{This table is split across pages}
%\tablefirsthead{%
%	\hline
%	\multicolumn{1}{|c}{\tbsp Symbolr} &
%	\multicolumn{1}{c}{Definition} & \\
%	\hline}
%\tablehead{%
%	\hline
%	\multicolumn{2}{|l|}{\small\sl continued from previous page}\\
%	\hline
%	\multicolumn{1}{|c}{\tbsp Symbolr} &
%	\multicolumn{1}{c}{Definition} & \\
%	\hline}
%\tabletail{%
%	\hline
%	\multicolumn{2}{r}{\small\sl continued on next page}\\
%	\hline}
%\tablelasttail{\hline}
%\bottomcaption{This table is split across pages}
%\tablefirsthead{%
%	\hline
%	\multicolumn{1}{c}{\tbsp Symbolr} &
%	\multicolumn{1}{c}{Definition} & \\
%	\hline}
%\tablehead{%
%	\hline
%	\multicolumn{2}{l}{\small\sl continued from previous page}\\
%	\hline
%	\multicolumn{1}{c}{\tbsp Symbolr} &
%	\multicolumn{1}{c}{Definition} & \\
%	\hline}
%\tabletail{%
%	\hline
%	\multicolumn{2}{r}{\small\sl continued on next page}\\
%	\hline}
%\tablelasttail{\hline}
%\bottomcaption{This table is split across pages}
%\begin{supertabular}{c@{\hspace{6.5mm}}c}
%			X &  Trips travel down the Big Long River each year during a six month period \\
%	Y &  Camp sites on the Big Long River \\
%	S &  The total length of the river \\
%	${v_1}$ &  The speed of oarpowered rubber rafts \\
%	${v_2}$ &  The speed of motorized boats \\
%	${p_i}$ &  Travel options \\
%	$U_i^k$ &  Unit camp matrix \\
%	${C_i}$ &  Camping matrix \\
%	$A$ &  Total camp camp matrix \\
%	$J$ &  The launch date of a given tour group \\
%	${n_i}$ &  The number of days the  th travel option lasts \\
%	$\lambda $ &  the ith dam in a series of small dams \\
%	\hline 
%\end{supertabular}
%\end{center}
\begin{table}[H]
	\begin{center}
	\caption{\label{tab:Symbols_total}Symbols and Definitions}
	\begin{tabular}{c p{13cm}}
	\toprule
		Symbol  & Definition \\
		\midrule
		X &  Trips travel down the Big Long River each year during a six month period \\
		Y &  Camp sites on the Big Long River \\
		S &  The total length of the river \\
		${v_1}$ &  The speed of oarpowered rubber rafts \\
		${v_2}$ &  The speed of motorized boats \\
		${p_i}$ &  Travel options \\
		$U_i^k$ &  Unit camp matrix \\
		${C_i}$ &  Camping matrix \\
		$A$ &  Total camp camp matrix \\
		J &  The launch date of a given tour group \\
		${n_i}$ &  The number of days the  th travel option lasts \\

		$\lambda $ &  the ith dam in a series of small dams \\
	\bottomrule
	\end{tabular}
\end{center}
\end{table} %符号设置
\section{Models}

\begin{table}[H]
	\centering
	\caption{Global rankings in 2014, 2015, 2016, 2017 Change in total headcount usage in the top 23 languages}
	\begin{tabular}{llrrrr}
		\toprule
		\multicolumn{1}{l}{Number} & Language & 2014  & 2015  & 2016  & 2017 \\
		\midrule
		1     & Chinese,Mandarin & 1052.535 & 1070.966 & 1073.525 & 1076.734 \\
		2     & Spanish & 441.7462 & 453.7809 & 466.8536 & 480.4602 \\
		3     & English & 816.9127 & 903.37 & 911.6737 & 917.9667 \\
		4     & Hindi & 365.4672 & 365.735 & 366.4932 & 367.5688 \\
		5     & Portuguese & 196.7223 & 198.126 & 200.7203 & 202.5036 \\
		6     & Bengali & 184.1833 & 186.2771 & 183.2275 & 179.4995 \\
		7     & Russian & 189.6124 & 192.1039 & 194.2471 & 197.3681 \\
		8     & Japanese & 124.9867 & 125.1537 & 125.3209 & 125.5303 \\
		9     & Javanese & 77.36284 & 76.93927 & 76.51803 & 75.99483 \\
		10    & German,Standard & 146.0954 & 146.6092 & 147.6057 & 148.2428 \\
		11    & Chinese,Wu & 73.22911 & 72.48929 & 71.75694 & 70.85228 \\
		12    & Korean & 117.8224 & 117.8665 & 117.9106 & 117.9658 \\
		13    & French & 126.9646 & 153.8068 & 154.9492 & 155.939 \\
		14    & Telugu & 81.37525 & 81.70919 & 82.14315 & 82.18562 \\
		15    & Marathi & 71.41339 & 71.47918 & 71.54502 & 71.62742 \\
		16    & Turkish & 58.45415 & 58.32959 & 58.2053 & 58.05032 \\
		17    & Tamil & 73.58453 & 73.5565 & 73.52849 & 73.4935 \\
		18    & Vietnamese & 67.34071 & 67.26303 & 67.18545 & 67.0886 \\
		19    & Urdu  & 153.8177 & 153.5716 & 153.3494 & 153.0674 \\
		20    & Italian & 64.94941 & 64.9459 & 64.94238 & 64.93799 \\
		\bottomrule
	\end{tabular}%
	\label{tab:addlabel}%
\end{table}%

\begin{figure}[H]
	\centering
	\includegraphics[width=1\linewidth]{figures/next50number}
	\caption{The number of speakers in major languages over the next 50 years}
	\label{fig:next50number}
\end{figure}

\begin{figure}[h]
	\centering
	\includegraphics[width=1\linewidth]{figures/percent}
	\caption{}
	\label{fig:percent}
\end{figure}


\begin{figure}[H]
	\centering
	\includegraphics[width=1\linewidth]{figures/english}
	\caption{}
	\label{fig:english}
\end{figure}


\begin{figure}[h]
	\centering
	\includegraphics[width=1\linewidth]{figures/Spanish}
	\caption{}
	\label{fig:spanish}
\end{figure}



\begin{figure}
	\centering
	\includegraphics[width=1\linewidth]{figures/chart}
	\caption{Language Influence Index Structure chart}
	\label{fig:chart}
\end{figure}

% Table generated by Excel2LaTeX from sheet 'Sheet1'
\begin{table}[H]
	\centering
	\caption{Language ranking comparison}
	\begin{tabular}{l p{2.8cm}p{2.8cm}p{2.8cm}p{2.8cm}}
		\toprule
		\multicolumn{1}{l}{Rank} & native speakers(2017 year) & native speakers(2067 year) & total speakers(2017 year) & total speakers(2067 year) \\
		\midrule
		1     & Mandarin & Mandarin & Mandarin & English \\
		2     & Spanish & Spanish & English & Mandarin \\
		3     & English & English & Hindustani & Spanish \\
		4     & Hindustani & Hindustani & Spanish & Hindustani \\
		5     & Arabic & Portuguese & Arabic & Arabic \\
		6     & Bengali & Arabic & Malay & Portuguese \\
		7     & Portuguese & Bengali & Russian & Russian \\
		8     & Russian & Russian, & Bengali & French, \\
		9     & Punjabi & Japanese & Portuguese & German,Standard \\
		10    & Japanese & German Standard. & French & Malay \\
		\bottomrule
	\end{tabular}%
	\label{tab:addlabel}%
\end{table}%


% Table generated by Excel2LaTeX from sheet 'Sheet2'
\begin{table}[H]
	\centering
	\caption{ The second language is mainly used in countries / regions}
	\begin{tabular}{l|l|p{4cm} |p{4cm}}
		\toprule
		language & Asian & Europe & Oceania \\
		\midrule
		Chinese &       &       & Australian \\
			\midrule
		English &       & Sweden, France, Iceland, Poland &  \\	\midrule
		spanish &       &       &  \\	\midrule
		Arabic &       &       &  \\	\midrule
		Russian &       &Ukraine, Kazakhstan, Uzbekistan,Turkmen&  \\	\midrule
		Bengalese & India&       &  \\	\midrule
		Portuguese &       &       &  \\	\midrule
		French &       &       &  \\	\midrule
		Hausa &       &       &  \\	\midrule
		Turkish &       & Austria,Germany, Bulgaria &  \\	\midrule
		Italian &       &       &  \\
		\bottomrule
		\toprule
		language & North America& South America &  Africa \\
		\midrule
		Chinese &       &       &  \\	\midrule
		English & Central America & & Egypt,Sudan ,Cameroon ,Cameroon \\	\midrule
		spanish & \multicolumn{1}{l}{America} & Brazil &  \\	\midrule
		Arabic &       &       & Ethiopia ,Chad ,Somalia, South Sudan \\	\midrule
		Russian &       &       &  \\	\midrule
		Bengalese &       &       &  \\	\midrule
		Portuguese &       &       &  \\	\midrule
		French & America&       & Mozambique \\	\midrule
		Hausa &       &       & the Nile ,Nigeria \\	\midrule
		Turkish &       &       &  \\	\midrule
		Italian &       & Argentina & Algeria \\	\midrule
		\bottomrule
	\end{tabular}%
	\label{tab:addlabel}%
\end{table}%


% Table generated by Excel2LaTeX from sheet 'Sheet2'
\begin{table}[H]
	\centering
	\caption{012-2013 scores of some languages and rankings}
	 \scalebox{0.87}[0.87]{%
	\begin{tabular}{|p{4.11em}|r|r|r|r|r|r|}
		\toprule
		\multicolumn{1}{|c|}{\multirow{2}[4]{*}{\backslashbox[0pt][l]{Language}{Time}}} & \multicolumn{2}{c}{2012} & \multicolumn{2}{c}{2013} & \multicolumn{2}{c|}{2014} \\
		\cmidrule{2-7}    \multicolumn{1}{|c|}{} & \multicolumn{1}{p{4.11em}|}{Score} & \multicolumn{1}{p{4.11em}|}{Rank} & \multicolumn{1}{p{4.055em}|}{Score} & \multicolumn{1}{p{4.055em}|}{Rank} & \multicolumn{1}{p{4.055em}|}{Score} & \multicolumn{1}{p{4.055em}|}{Rank} \\
    \midrule
	English & 0.853  & 1     & 0.853  & 1     & 0.752  & 1 \\
	\midrule
	Chinese & 0.416  & 2     & 0.401  & 2     & 0.318  & 2 \\
	\midrule
	French & 0.301  & 5     & 0.302  & 4     & 0.200  & 7 \\
	\midrule
	Spanish & 0.398  & 3     & 0.401  & 3     & 0.307  & 3 \\
	\midrule
	Arabic & 0.278  & 7     & 0.275  & 7     & 0.202  & 6 \\
	\midrule
	Russian & 0.226  & 8     & 0.224  & 8     & 0.140  & 9 \\
	\midrule
	German & 0.280  & 6     & 0.276  & 6     & 0.268  & 5 \\
	\midrule
	Japanese & 0.179  & 9     & 0.180  & 9     & 0.153  & 8 \\
	\midrule
	Portuguese & 0.145  & 10    & 0.142  & 10    & 0.137  & 10 \\
	\midrule
	Hindi & 0.305  & 4     & 0.301  & 5     & 0.301  & 4 \\
	\bottomrule
	\end{tabular}%
}
	\label{tab:012}%
\end{table}%



% Table generated by Excel2LaTeX from sheet 'Sheet2'
\begin{table}[H]
	\centering
	\caption{Change of language performance rankings every decade}
	\begin{tabular}{|p{5.22em}|r|r|r|r|r|r|}
		\toprule
		\multicolumn{1}{|r|}{\backslashbox[0pt][l]{Language}{Year}} & 2017  & 2027  & 2037  & 2047  & 2057  & 2067 \\
		\midrule
		English & 1     & 1     & 1     & 1     & 1     & 1 \\
		\midrule
		Chinese & 2     & 2     & 2     & 2     & 2     & 2 \\
		\midrule
		French & 3     & 3     & 4     & 4     & 4     & 4 \\
		\midrule
		Spanish & 4     & 4     & 3     & 3     & 3     & 3 \\
		\midrule
		Arabic & 5     & 5     & 5     & 5     & 6     & 6 \\
		\midrule
		Russian & 6     & 6     & 6     & 6     & 5     & 5 \\
		\midrule
		German & 7     & 7     & 7     & 7     & 7     & 8 \\
		\midrule
		Japanese & 8     & 9     & 9     & 10    & 10    & 10 \\
		\midrule
		Portuguese & 9     & 8     & 8     & 8     & 9     & 9 \\
		\midrule
		Hindi & 10    & 10    & 10    & 9     & 8     & 7 \\
		\bottomrule
	\end{tabular}%
	\label{tab:Change}%
\end{table}%


 % 模型
%\section{Conclusions}
According to our model, the trend of language development can be reflected in two aspects: the number of language speakers and the regional differences in languages.The location of the new international office is based on the key factors: the effectiveness of the language.
\par In the part of prediction of the trend of languages,we can obtain the change of the total number of speakers of the mainstream languages in the world(Shown in Fig.1), the variations of the number of native speakers in the mainstream languages(Shown in Fig.2),languages of top ten(both native speakers and total speakers) and their ranks in fifty years(see table3).What's more, We find that ranked currently in the top ten languages in the future rankings remain in the top ten, but the specific rankings changethe total number of speakers of English exceeds the number of speakers of Chinese in 2045, when the total population of the former reach 1,432.163 million.As for the part of geographical distribution of language speakers, we obtain that the Arabic speakers account for 50.13\% of the total Arabic speaker in 2048.Additionally, the most powerful growth is in Spanish.
\par In the part of siting in international offices,we build an evaluation model of language effectiveness. Based on this model, we examine the effectivenesse of the mainstream language over time. Further, we determine the language zones to be selected for the new international office based on the performance ranking of the language.Based on the results of our modeling, we find no significant differences in the rankings of language performance in the short term and long term,so there is no change in the construction plan for the new office.When we assume that we can capture the penetration of regional Internet and that the multinational companies have web services.We obtain the higher the Internet penetration rate in a given area, the lower the need to have an office there.Therefore, we propose to reduce the number of offices in developed regions and consider setting up offices in areas with great potential such a some areas of Africa % 结论
\section{Evaluate of the Mode} % 灵敏度分析
\section{Conclusions}
According to our model, the trend of language development can be reflected in two aspects: the number of language speakers and the regional differences in languages.The location of the new international office is based on the key factors: the effectiveness of the language.
\par In the part of prediction of the trend of languages,we can obtain the change of the total number of speakers of the mainstream languages in the world(Shown in Fig.1), the variations of the number of native speakers in the mainstream languages(Shown in Fig.2),languages of top ten(both native speakers and total speakers) and their ranks in fifty years(see table3).What's more, We find that ranked currently in the top ten languages in the future rankings remain in the top ten, but the specific rankings changethe total number of speakers of English exceeds the number of speakers of Chinese in 2045, when the total population of the former reach 1,432.163 million.As for the part of geographical distribution of language speakers, we obtain that the Arabic speakers account for 50.13\% of the total Arabic speaker in 2048.Additionally, the most powerful growth is in Spanish.
\par In the part of siting in international offices,we build an evaluation model of language effectiveness. Based on this model, we examine the effectivenesse of the mainstream language over time. Further, we determine the language zones to be selected for the new international office based on the performance ranking of the language.Based on the results of our modeling, we find no significant differences in the rankings of language performance in the short term and long term,so there is no change in the construction plan for the new office.When we assume that we can capture the penetration of regional Internet and that the multinational companies have web services.We obtain the higher the Internet penetration rate in a given area, the lower the need to have an office there.Therefore, we propose to reduce the number of offices in developed regions and consider setting up offices in areas with great potential such a some areas of Africa % 结论
\section{Strengths and weaknesses}
\subsection{Strengths}
\begin{itemize}
	\item \textbf{Improve the existing model}
\newline Existing indicators of linguistic influence assessment model change and the evaluation of linguistic effectiveness that is more suitable for solving this problem are improved. To quantify the linguistic influence of the abstract description to a fraction of [0,1] and calculate the language score to provide a reasonable basis for the new international office location
	\item \textbf{Good flexibility}
\newline To study this problem, two main models are established: the non-equidistant GM (1,1) prediction model and the language performance evaluation model (LII). The former model can be used for other forecasting problems. The latter model can be extended to the evaluation problem. The model is more flexible and can be transformed into a solution to other practical problems
	\item \textbf{Justifiability}
\newline When selecting the evaluation indicators, we start from the five dimensions of geography, economy, total number of languages spoken in one language (inc.l1 \& l2), information output and quality of the language, and the situation in which the international organizations use the language. The selected index contains more information to ensure the rationality of our evaluation model.	
	
\end{itemize}

\subsection{weaknesses}

\begin{itemize}
	\item \textbf{Inaccuracy} \newline
	In the first part of the forecast of the development trend of the global language, due to the time factor analysis did not use multivariate analysis, nor did it fully consider the background of the factors. Based on the actual data we found, we used the predictive model to predict the data, which is a little bit different from the actual situation
	\item \textbf{Simplifying assumptions}
	\newline In the establishment of a population migration model, the simplification assumption assumes that only the net migration population is considered, the gravitation analysis for the population movements is not enough, and the geographical distribution and proliferation of the second language caused by the actual population flows in terms of the number of refugees, economic factors and policy factors are not tapped.
	\item \textbf{Lack of data}
	\newline 
	The census and statistics of the global language population are extremely difficult and complex tasks. This question is required to collect a lot of data, the subject only gives a year of data, you must have nearly 10 years of data to make long-term forecasts. We collect less valid data, which leads to inaccurate results of the forecasting model.
\end{itemize}
\section{Future Improvements}

\begin{itemize}
\item Since we did not fully consider the background factors given in the subject when we set up the forecasting model, we only make a prediction based on the data of the linguistic population itself, and the error between this forecast and the actual situation is relatively large. The main reason is that less than the relevant data cannot be collected. If we can get the historical data of 5-10 years from the relevant factors and we can correct the predicted values through the relevant factor analysis, then the result of our model will be more accurate.\\
\item For the change of linguistic population on geographical distribution, we may have some limitations to simplify the analysis and to study only the changes of the net migration population under the research of the state. We believe that we can also analyze the language flow direction from the discrepancy or gravity of any two languages to establish the "language distance" model or the gravity model. Thus, we can draw a directed graph of geographical distribution of the number of second-language mainstream in the world and more vividly analyze the impact of the mode of population migration on geographical distribution of languages.
\end{itemize} %进一步提高
\input{paper/Memorandum} % 说明书
% =================================================================
%引用别人的成果或其他公开的资料(包括网上资料)必须按下列方式在正文引用处和参考文献中明确列出。正文引用处用方括号标示参考文献的编号,如[1]、[3]等;引用书籍必须指出页码。参考文献按正文中的引用次序列出,其中书籍的表述方式为:
%[编号] 作者,书名,出版地:出版社,出版年。
%参考文献中期刊杂志论文的表述方式为:
%[编号] 作者,论文名,杂志名,卷期号:起止页码,出版年。
%参考文献中网上资源的表述方式为:
%[编号] 作者,资源标题,网址,访问时间(年月日)。

%\begin{thebibliography}{99}
%	\addcontentsline{toc}{section}{References} % 在目录中添加参考文献
%		
%	\bibitem{1}\url{http://canyonx.com/trip_lengths_logistics.php}
%	\bibitem{2}Shang Pengchao, Zhang Cong, Ren Qingfeng, et al.Arrangement of drifting travel plan [J] .Journal of Yan'an University (Natural Science Edition), 2012, 31 (3): 48-50.
%	Addison-Wesley Publishing Company, 1986.
%	\bibitem{3}XU Dao, CAO Xiaoyu, GENG Jianghua.Application of 0-1 Integer Programming Model to Rafting Travel Schedule [J]. Science Technology Information, 2012 (27): 158-158.
%\end{thebibliography}
\cite{hegazy1999evosite}
Some excellent books, for example, \cite{Lamport1994} and \cite{Mittelbach2004} \ldots \upcite{tam2002site}
\bibliography{books}
\begin{appendices}
	
	\section*{Code appendix} % \section*{title} 不加入目录
	
	Here are simulation programmes we used in our model as follow.\\
	
	
	
	\textbf{\textcolor[rgb]{0.98,0.00,0.00}{MATLAB code of non-equal spacing of GM (1,1) :}}
	\lstinputlisting[language=matlab]{./code/L2.m}

	\section*{Charts and Figures}
	% Table generated by Excel2LaTeX from sheet 'Sheet2'
	\begin{table}[H]
		\centering
		\caption{ The second language is mainly used in countries / regions}
		\begin{tabular}{llp{4cm}p{4cm}}
			\toprule
			language & Asian & Europe & Oceania \\
			\midrule
			Chinese &       &       & Australian \\
			\midrule
			English &       & Sweden, France, Iceland, Poland &  \\	\midrule
			spanish &       &       &  \\	\midrule
			Arabic &       &       &  \\	\midrule
			Russian &       &Ukraine, Kazakhstan, Uzbekistan,Turkmen&  \\	\midrule
			Bengalese & India&       &  \\	\midrule
			Portuguese &       &       &  \\	\midrule
			French &       &       &  \\	\midrule
			Hausa &       &       &  \\	\midrule
			Turkish &       & Austria,Germany, Bulgaria &  \\	\midrule
			Italian &       &       &  \\
			\bottomrule
			\toprule
			language & North America& South America &  Africa \\
			\midrule
			Chinese &       &       &  \\	\midrule
			English & Central America & & Egypt,Sudan ,Cameroon ,Cameroon \\	\midrule
			spanish & \multicolumn{1}{l}{America} & Brazil &  \\	\midrule
			Arabic &       &       & Ethiopia ,Chad ,Somalia, South Sudan \\	\midrule
			Russian &       &       &  \\	\midrule
			Bengalese &       &       &  \\	\midrule
			Portuguese &       &       &  \\	\midrule
			French & America&       & Mozambique \\	\midrule
			Hausa &       &       & the Nile ,Nigeria \\	\midrule
			Turkish &       &       &  \\	\midrule
			Italian &       & Argentina & Algeria \\	\midrule
			\bottomrule
		\end{tabular}%
		\label{tab:SixContinents}%
	\end{table}%

% Table generated by Excel2LaTeX from sheet 'L1 (Native) Speakers'
\begin{table}[H]
	\centering
	\caption{L1 (Native) Speakers}
	\begin{tabular}{rlrrrrrr}
		\multicolumn{1}{l}{Rank} & Language & 2015  & 2009  & 2005  & 2000  & 1996  & 1961 \\
		1     & Chinese, Mandarin  & 848   & 845   & 873   & 874   & 885   & 460 \\
		2     & Spanish  & 399   & 329   & 322.3 & 322.2 & 226   & 140 \\
		3     & English  & 335   & 328   & 312.5 & 308.2 & 305.6 & 250 \\
		4     & Hindi  & 260   & 259.6 & 259.5 & 259.5 & 182   & 65 \\
		5     & Portuguese  & 203   & 178   & 177.5 & 176   & 170   & 75 \\
		6     & Bengali  & 189   & 193.1 & 193.5 & 182.7 & 168.1 & 75 \\
		7     & Russian  & 166   & 144   & 143.2 & 155.3 & 155   & 130 \\
		8     & Japanese  & 128   & 122   & 122.4 & 122.6 & 122.2 & 95 \\
		9     & Javanese  & 84.3  & 84.6  & 75.5  & 75.5  & 75.5  & 45 \\
		10    & German, Standard  & 85.6  & 90.1  & 95.2  & 95.1  & 98    & 100 \\
		11    & Chinese, Wu  & 80.1  & 78.5  & 77.1  & 74.4  & 70.8  & 50 \\
		12    & Korean  & 77.2  & 69.5  & 67.1  & 67    & 65.5  & 46 \\
		13    & French  & 75.9  & 67.8  & 64.8  & 64.7  & 64    & 65 \\
		14    & Telugu  & 74    & 69.8  & 69.7  & 69.7  & 66.5  & 37 \\
		15    & Marathi  & 71.8  & 68.1  & 68    & 68    & 64.8  & 31 \\
		16    & Turkish  & 70.9  & 71.1  & 50.5  & 50.7  & 50.2  & 25 \\
		17    & Tamil  & 68.8  & 65.7  & 66    & 66    & 62    & 32 \\
		18    & Vietnamese  & 67.8  & 68.6  & 67.4  & 67.7  & 66.9  & 24 \\
		19    & Urdu  & 64    & 60.6  & 60.5  & 60.3  & 56.6  & 75 \\
		20    & Italian  & 63.8  & 61.7  & 61.5  & 61.4  & 61.4  & 55 \\
	\end{tabular}%
	\label{tab:addlabel}%
\end{table}%

	
\end{appendices} 
% =================================子文件夹设置====================================
\end{document}
